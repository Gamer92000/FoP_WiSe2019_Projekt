\documentclass{article}
\usepackage[utf8]{inputenc}
\usepackage{xcolor}
\usepackage{colortbl}
\usepackage{graphicx}
\usepackage{tikz}
\usepackage{lmodern}
\usepackage{amssymb,amsmath}
\usepackage{ifxetex,ifluatex}
\usepackage{fixltx2e} % provides \textsubscript
\usepackage{ngerman}
\usepackage{float}
\usepackage{caption}
\usepackage{subcaption}


\PassOptionsToPackage{hyphens}{url} % url is loaded by hyperref
\usepackage[unicode=true]{hyperref}

\usepackage{color}
\usepackage{fancyvrb}

\usepackage{mathtools}

\usepackage[a4paper, margin=3cm]{geometry}

\usepackage{listings}
\lstset{numbers=left, numberstyle=\tiny, numbersep=5pt}
\lstset{language=Java}

\setlength{\parindent}{0pt}

\DeclarePairedDelimiter\floor{\lfloor}{\rfloor}
\newcommand*{\qed}{\hfill\ensuremath{\blacksquare}\\}%

% set default figure placement to htbp
\makeatletter

\title{FOP Projekt Wintersemester 2019/20}
\author{Yannik Sebastian Hayn, Julian Imhof,\\ Erik Prescher, Lennart Schmidt}
\makeatletter
\renewcommand*{\ps@plain}{%
  \let\@mkboth\@gobbletwo
  \let\@oddhead\@empty
  \def\@oddfoot{%
    \reset@font
    \hfil
    \thepage
    % \hfil % removed for aligning to the right
  }%
  \let\@evenhead\@empty
  \let\@evenfoot\@oddfoot
}
\usepackage{listings}
\lstset{language=Java,
  showspaces=false,
  showtabs=false,
  breaklines=true,
  showstringspaces=false,
  breakatwhitespace=true,
  basicstyle=\ttfamily
}
\makeatother
\pagestyle{plain}

\begin{document}
\maketitle
\section{Aufgabe 6.1.5 Theorie}
\subsection{Azyklischer Graph}
Da nur ein gerichteter Graph azyklisch sein kann, der hier gegebene Graph allerdings nicht gerichtet ist, kann er folglich auch nicht azyklisch sein.\\
 Im Folgenden gehen wir allerdings davon aus, dass auch ein nicht gerichteter Graph azyklisch sein k\"onne. Dabei nehmen wir an, dass ein Zyklus aus mindestens drei Nodes bestehen muss und der Zyklus \"uber $n - 1$ Kanten durchlaufen werden k\"onnen muss, ohne dabei zwei mal \"uber eine Kante zu gehen, wobei $n = \textit{Anzahl der Nodes}$ ist.\\
Im Folgenden zeigen wir, dass der Graph zyklisch sein kann.\\

Betrachten wir einmal die einfachste Idee, um einen Zyklus zu erzeugen: Ein Kreisverkehr aus mehreren Stra{\ss}enkarten.
\begin{figure}[H]
	\centering
	\begin{subfigure}{.5\textwidth}
		\centering
		\includegraphics[width=0.9\textwidth]{Zyklischer_Graph.png}
		\caption{v.l.n.r. Pl\"attchen: V, J, J, J}
	\end{subfigure}%
	\begin{subfigure}{.5\textwidth}
		\centering
		\includegraphics[width=0.9\textwidth]{Zyklischer_Graph1.png}
		\caption{zugeh\"origer Graph}
	\end{subfigure}
	\caption{Eine einfache M\"oglichkeit mit wenigen Pl\"attchen viele Zyklen zu erzeugen}
\end{figure}
Wie man sehen kann, sind durch diese 4 Karten bereits 3 Zyklen entstanden.\\ Der Graph ist somit nicht azyklisch.\qed

Wenn man diese Idee ein bisschen weiter f\"uhrt, kommt man relativ schnell zu dem Schluss, dass der Graph in den allermeisten F\"allen sp\"atestens bei Spielende zyklisch sein wird.

\subsection{Zusammenhangskomponenten}
F\"ur jedes Pl\"attchen $p$ gilt, dass es theoretisch maximal 9 Teilmengen von verschiedenen Zusammenhangskomponenten haben kann.
\begin{figure}[H]
	\centering
	\includegraphics[width=0.3\textwidth]{aufgabe2_0.png}
	\caption{zeigt alle maximal m\"oglichen Zusammenhangskomponenten eines einzelnen Pl\"attchens}
\end{figure}
F\"ur $k$ nicht verbundene Pl\"attchen gilt also:
\begin{align*}
n &\leq 9 * k\\
\end{align*}
Nur die Knoten eines neu angelegten Pl\"attchens k\"onnen eine neue Zusammenhangskomponente bilden, wenn sie nicht Teil einer Kante mit einem bereits gelegten Pl\"attchen sind. Da ein neues Pl\"attchen aber nur an mindestens einem alten Pl\"attchen angelegt werden kann, gilt, dass mindestens drei Knoten an einer gemeinsamen Kante liegen, und somit maximal 6 neue Zusammenhangskomponenten existieren k\"onnen. 
\begin{figure}[H]
	\centering
	\includegraphics[width=0.5\textwidth]{aufgabe2_1.png}
	\caption{zeigt, dass es nur maximal 6 neue Zusammenhangskomponenten geben kann}
\end{figure}
Somit gilt, wenn man ein neues Pl\"attchen an nur einem alten anlegt:
\begin{align*}
n &\leq 9 + 6*(k-1)
\end{align*}
Legt man ein neues Pl\"attchen hingegen an 2 oder mehr Pl\"attchen an, so ergeben sich immer weniger als $6$ potentielle neue Zusammenhangskomponenten, wie man Abbildung 4 entnehmen kann.
\begin{figure}[H]
	\centering
	\begin{subfigure}{.5\textwidth}
		\centering
		\includegraphics[width=0.9\textwidth]{aufgabe2_2.png}
	\end{subfigure}%
	\begin{subfigure}{.5\textwidth}
		\centering
		\includegraphics[width=0.9\textwidth]{aufgabe2_3.png}
	\end{subfigure}\\
	\begin{subfigure}{.5\textwidth}
		\centering
		\includegraphics[width=0.9\textwidth]{aufgabe2_4.png}
	\end{subfigure}%
	\begin{subfigure}{.5\textwidth}
		\centering
		\includegraphics[width=0.9\textwidth]{aufgabe2_5.png}
	\end{subfigure}
	\caption{Alle M\"oglichkeiten ein neues Pl\"attchen an mehr als 2 bereits gelegte Pl\"attchen anzulegen}
\end{figure}
Somit gilt die oben ermittelte Ungleichung also f\"ur alle an das Startpl\"attchen angelegte Pl\"attchen.
\begin{align*}
n &\leq 9 + 6*(k-1)\\
\Leftrightarrow n &\leq 9 + 9*(k-1)\\
\Leftrightarrow n &\leq 9*(k-1+1)\\ 
\Leftrightarrow n &\leq 9k\\
\end{align*}
Somit zeigt sich, dass die gegebene Ungleichnug nicht die Pr\"aziseste, allerdings korrekt ist.
\qed

\newpage

\subsection{Laufzeitkomplexit\"at von calculatePoints()}
Sei der Graph $G = (V,E)$ gegeben.\\
In der Methode \textit{calculatePoints(FeatureType type, State state)} werden alle Nodes und Edges im Graph der Reihe nach bearbeitet, um anschlie$\ss$end f\"ur jedes zusammenh\"angende Konstrukt im Spiel die Punkte zu errechnen. Die Reihenfolge ist dabei abh\"angig von den Konstrukten, was sich jedoch nicht weiter bedeutend auf die Laufzeit dieser Berechnung auswirkt.\\
Zu jeder Node werden gleicherma$\ss$en alle Edges auf Zusammenh\"angigkeit untersucht. Vereinfacht kann man die Komplexit\"at mit der der folgenden Funktion beschreiben:
\begin{lstlisting}
for (node : graph)
  for (edge : graph)
    [...]
\end{lstlisting}
Das entspricht einer Komplexit\"at von $\mathcal{O}(|V|*|E|)$.\\
Hinzu kommen die Berechnungen zur Bewertung. Die H\"aufigkeit dieser ist allein abh\"angig von der Anzahl an Konstrukten.\\
Sei $|S|$ also die Anzahl an Konstrukten. Dann ist die Komplexit\"at beschreibbar durch
\begin{align*}
&\mathcal{O}(|V|*|E|+|S|)\\
= \ &\mathcal{O}(|E|*|V|)
\end{align*}
\qed

\section{Dokumentation}
\subsection{calculatePoints}
Die Methode operiert als eine Queue, in die zunächst eine zufällige Node aus der Liste aller Nodes zum entsprechenden Typ geladen wird. Über diese können anschließend mit dem Graph alle verbundenen Nodes, sowie die relevanten Informationen, wie Tiles, Meaples, sowie für Burgen Pennants ermittelt werden. Die verbundenen Nodes werden in die Queue geladen, die ursprüngliche Node daraus entfernt und die weiteren Informationen vorerst in Listen gespeichert. Weiterhin wird zunächst davon ausgegangen, das aktuelle Konstrukt abgeschlossen ist. Sobald eine Node gefunden wird, die sich am Rand befindet, aber keine Tiles verbindet, wird der status auf unabgeschlossen gesetzt. Sind zu einem Konstrukt alle Nodes gefunden worden, d.h. die Queue ist leer, so werden abhängig vom Spielstatus und anhand der ermittelten Informationen die Punkte entsprechend der Spielregeln berechnet und gutgeschrieben.\\
Bei Wiesen gibt es noch einen Sonderfall. Hier wird jede Node auf anliegende Burgen-Nodes überprüft. Gehört diese zu einer abgeschlossenen Burg, so werden 3 Punkte an den Besitzer der Wiese vergeben.

\subsection{Highscore View}
In der Highscore View wurde die Schriftart der Highscores auf Celtic geändert, damit das Spiel einen einheitliches Gesamtaussehen und einen durchgängiges Design hat. Außerdem wurden die ersten 3 Highscores in Gold, Silber, Bronze eingefärbt und anschließend mit einem Gradient versehen, um die Anzeige dynamischer zu gestalten. Ab der 3. Zeile wechseln sich Weiß und ein helles Grau ab um die Zeilen besser verfolgen zu können. Die Horizontalen Linien wurden dafür ausgeschaltet.

\subsection{Mission}

\subsection{Computergegner}

\end{document}
